\section{DECISION MAKING}
In the team context, especially for our project, decision-making is essential for progress and achieving our goals. Given our collective inexperience, we often face scenarios where the correct approach is unclear, with each member potentially advocating for different solutions. This diversity in problem-solving approaches necessitates a structured decision-making process to navigate through project milestones, deadlines, and any arising conflicts or delays. \par
\smallskip
To address this, we've established a voting system categorized into three types of decisions: routine, decisions of change, and significant decisions. Routine decisions cover everyday project management tasks, such as assigning roles for meetings or document management, and require a simple majority (at least 4 members in agreement) for resolution.\par
\smallskip
Decisions for change are invoked during disagreements or conflicts, necessitating not only a majority vote but also compelling arguments from team members to support their viewpoints. This category includes resolving code conflicts, making changes to the project's design, or reassigning tasks among members. \par
\smallskip
Significant decisions, which could drastically alter the project's direction or structure (e.g., removing features, overhauling the project layout, or restarting the project), demand unanimous consent and strong justification from all team members due to their profound impact.\par
\smallskip
Finally, recognizing our limitations and the complexity of certain decisions, we value the input of our Teacher Assistant (TA). Consulting the TA for advice ensures informed decisions in critical scenarios, leveraging their experience to guide our project towards successful completion. This decision-making framework is designed to balance efficiency with thorough deliberation, ensuring that all team members contribute to steering the project in the right direction.\par
\smallskip