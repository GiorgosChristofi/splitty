% !TEX root =  ../report.tex
% \section{Adding Structure}

Avoid to over-structure your document, e.g., with extensive use of the various levels of \textbackslash{}section.
Instead, Latex offers a rich arsenal of tools that can help to structure documents

\smallskip
For a small jump in the argumentation, it is usually enough to use a \textbackslash{}smallskip (or \textbackslash{}medskip, \textbackslash{}bigskip).
Along these lines, avoid using \textbackslash{}\textbackslash{} in your document to break into new lines, simply use a "double newline" in the source document.

Like this. As you see, the new paragraph will be indented to highlight the start of the next paragraph.
If this indentation is a concern for you in your document, this is often a sign that your paragraphs are too short.



%%%
\paragraph{Named Paragraphs}
For bigger jumps or to add a bit more context, using \textbackslash{}paragraph is probably the easiest way to make it clear to a reader that you want to start a new thought.


%%%
\paragraph{Lists}

You very often you have to list various options or alternatives in your written texts.
You can use an itemize (or enumerate) environment to create a list:

\begin{itemize} % or: \begin{enumerate}
\item This is a bullet point.
\item This is a long bullet point that breaks a line. Please note that these lists should only be used sparsely, the biggest part of your report should be elaborated text.
\end{itemize} % or: \end{enumerate}

Another option for structuring your document \emph{in-the-small} is to use a description list.
In this type of list, you define a key, and the corresponding explanation of the key.

\begin{description}
\item [First Key] This is the elaboration.
\item [A Second Key] This is the second elaboration. Please note how Latex takes care of aligning the text in line 2+.
\end{description}

All these structural element can help with adding structure to the text and making it easier to read.
Please note how much less of an interruption they are for the overall flow of the text, compared to regular section.

Latex might be scary to newcomers, but knowing a few basic formatting commands is all that is necessary to get started.
For advanced concepts, the \emph{documentation of the Overleaf platform}\footnote{\url{https://www.overleaf.com/learn}} is usually a good start.
